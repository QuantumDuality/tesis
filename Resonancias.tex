\chapter{Resonancias en mecánica cuántica}

\section{Procesos de dispersión en mecánica cuántica}

En mecánica cuántica, un proceso de dispersión puede ser analizado a partir de proponer la siguiente forma asintótica para la función de onda
\begin{equation}
	\psi_{\infty} = A \left[e^{i k z} + f(\theta)  \frac{e^{i k r}}{r} \right]. \label{FAD}
\end{equation}

Donde el primer término en los corchetes representa la onda incidente, mientras el segundo término representa la onda dispersada, por lo que $f$ es entonces la amplitud de la dispersión. Ya que en este caso se tiene un problema en coordenadas esféricas, (\ref{FAD})



