\chapter{Resonancias en mecánica cuántica}

\section{Ecuación de Schrödinger en coordenadas esféricas}

Los potenciales creados en la subsecciones anteriores corrsponden al caso tridimensional con una simetria esférica con un ptencial central. Por lo tanto, su ley dinámica general está dada por la ecuación de Schröinger

\begin{equation*}
-\nabla^2 \psi + V(r) \psi = E \psi \,\,\, \frac{\hbar^2}{2}= 1.
\end{equation*}

En este caso la función de onda depende de una variable radial $r$ y dos angulares $\theta, \phi$. El primer término del lado izquierdo toma entonces la forma

\begin{equation*}
-\nabla^2 \psi = \frac{1}{r^2} \frac{\partial}{\partial r} \left(r^2 \frac{\partial \psi}{\partial r} \right) + \frac{1}{r^2 \sin \theta} \frac{\partial}{\partial \theta} \left(\sin \theta \frac{\partial \psi}{\partial \theta} \right) +  \frac{1}{r^2 \sin^2 \theta} \frac{\partial^2 \psi}{\partial^2 \phi}
\end{equation*}



\section{Procesos de dispersión en mecánica cuántica}

En mecánica cuántica, un proceso de dispersión puede ser analizado a partir de proponer la siguiente forma asintótica para la función de onda
\begin{equation}
	\psi_{\infty} = A \left[e^{i k z} + f(\theta)  \frac{e^{i k r}}{r} \right]. \label{FAD}
\end{equation}

Donde el primer término en los corchetes representa la onda incidente, mientras el segundo término representa la onda dispersada, por lo que $f$ es entonces la amplitud de la dispersión. Ya que en este caso se tiene un problema en coordenadas esféricas, (\ref{FAD}) debe satisfacer la ecuación de Schrödinger 

\begin{equation}
-\frac{d^2\psi_{\infty}}{dr^2} + \frac{l(l+1)}{r^2}\psi_{\infty} = k^2 \psi_{\infty} , \,\,\, \frac{\hbar^2}{2m}=1, \,\, E=k^2
\end{equation}


