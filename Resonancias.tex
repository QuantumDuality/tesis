\chapter{Resonancias en mecánica cuántica}

\section{Ecuación de Schrödinger en coordenadas esféricas}

Los potenciales creados en la subsecciones anteriores corrsponden al caso tridimensional con una simetria esférica con un ptencial central. Por lo tanto, su ley dinámica general está dada por la ecuación de Schröinger

\begin{equation}
-\nabla^2 \psi + V(r) \psi = E \psi \,\,\, \frac{\hbar^2}{2}= 1. \label{ESCE}
\end{equation}

En este caso la función de onda depende de una variable radial $r$ y dos angulares $\theta, \phi$. El primer término del lado izquierdo toma entonces la forma

\begin{equation*}
-\nabla^2 \psi = \frac{1}{r^2} \frac{\partial}{\partial r} \left(r^2 \frac{\partial \psi}{\partial r} \right) + \frac{1}{r^2 \sin \theta} \frac{\partial}{\partial \theta} \left(\sin \theta \frac{\partial \psi}{\partial \theta} \right) +  \frac{1}{r^2 \sin^2 \theta} \frac{\partial^2 \psi}{\partial^2 \phi}.
\end{equation*}

Podemos entonces realizar una searación de variables proponiendo una solución de la forma $\psi(r, \theta, \phi)= R(r)\Theta(\theta)\Phi(\phi)$. Sustituyendo esta expresión junto con el laplaciano en (\ref{ESCE}), podemos desacoplar dicha ecuación a través de dos constantes de sepración $l(l+1)$ y $m$

\begin{eqnarray}
-\frac{d}{dr}\left(r^2 \frac{d R}{dr} \right) + [r^2 V(r) +  l (l + 1)]R &=& E R \label{ESR}
\\
\sin{\theta}\frac{d}{d\theta}\left(\sin{\theta}\frac{d\Theta}{d\theta} \right)+l(l+1)\sin^2{\theta} \Theta &=& m^2 \Theta
\\
\frac{d^2 \Phi}{d \theta^2} &=& -m^2 \Phi.
\end{eqnarray}

Las soluciones a las ecuaciones de las variables angulares son

\begin{eqnarray*}
	\Theta^m_l & = & A_{ml} P^m_l(\cos{\theta})
	\\
	\Phi_m =B_m  e^{i m \phi}
\end{eqnarray*}

Donde $m$ está restringido al intervalo $-l, ... ,l$. Además

\begin{equation*}
	P^m_l(x) = (1-x^2)^{|m|/2} \frac{d^{|m|}}{dx^{|m|}} P_l(x),
\end{equation*}

donde $P_l$ es el polinomio de Legendre de orden $l$. Normalmente, el producto $\Theta \Phi$ se esribe como una sola función ortonormal con respecto a $l$ y $m$, conocida como harmónico esférico, dada por
\begin{equation*}
Y^m_l(\theta,\phi) = \epsilon \left[ \frac{(2l + 1)(l - |m|!)}{4 \pi (l + |m|)!}\right]^{1/2} e^{i m \phi}P^m_l(\cos{\theta}),
\end{equation*}
donde $\epsilon=(-1)^m$ para $m \le 0$ y $\epsilon=1$ para $m \le 0$.

Finalmente, la solución correspondiente a la ecuación para la parte radial dependerá de la forma de $V(r)$.
Ya que en este trabajo se trata principalmente en analizar la parte continua del espectro en potenciales de corto alcance (o con un comportamiento asintótico equivalente), En la siguiente sección se explicará coomo analizar los procesos de dispersión para este tipo de potenciales.



\section{Dispersión en mecánica cuántica en tres dimensiones}

En mecánica cuántica, un proceso de dispersión en un potencial de corto alcance puede ser analizado a partir de proponer la siguiente forma asintótica para la función de onda
\begin{equation}
	\psi_{\infty} = A \left[e^{i k z} + f(\theta)  \frac{e^{i k r}}{r} \right]. \label{FAD}
\end{equation}

Donde el primer término en los corchetes representa la onda incidente, mientras el segundo término representa la onda dispersada, por lo que $f$ es entonces la amplitud de la dispersión. Ya que en este caso se tiene un problema en coordenadas esféricas, la parte radial de $\psi_\infty$ debe satisfacer la ecuación (\ref{ESR}), que en la región correspondiente toma la frma 

\begin{equation}
-\frac{d}{dr} \left(r^2 \frac{d R}{dr} \right) +  l (l + 1) R = E R
\end{equation}

Cuya solución general está dada por las funciones de Bessel esféricas

\begin{equation*}
\psi=A_k j_l(k r) + B n_l(kr), ~~~ k^2=E
\end{equation*}

Donde la función $n_l$ es irregular en el origen, por lo que $B=0$. Por otro lado, la forma asintótica de $j_l$ es

\begin{equation*}
j_l(k r) \approx \sqrt{\frac{2}{\pi k r}} \sin \left(kr-\frac{l \pi}{2} + \frac{\pi}{4}\right).
\end{equation*}

Podemos entonces usar la splución completa que incluye la parte angular para escribir (\ref{FAD}) de la forma de una combinación lineal 

\begin{equation*}
\psi_\infty(r,\theta) = \sum_{\infty}^{l=0}{A_k A_{ml} P^m_l(\cos{\theta})\sin \left(kr-\frac{l \pi}{2} + \frac{\pi}{4}\right).}
\end{equation*}